\documentclass[../DoAn.tex]{subfiles}
\begin{document}


\section{Đặt vấn đề}
\label{section:1.1}
Trong kỷ nguyên Công nghiệp 4.0, các doanh nghiệp tràn ngập khối lượng dữ liệu chưa từng có, hệ quả trực tiếp của những tiến bộ trong công nghệ kỹ thuật số và các hệ thống kết nối. Sự gia tăng dữ liệu này đã thúc đẩy các công ty đổi mới và áp dụng các phương pháp phức tạp để hiển thị và quản lý hiệu quả. Việc tích hợp phân tích dữ liệu lớn, điện toán đám mây và IoT (Internet of Things) đã cho phép các doanh nghiệp xử lý và trực quan hóa các tập dữ liệu phức tạp hiệu quả hơn bao giờ hết. Do đó, việc ra quyết định dựa trên dữ liệu đã trở thành nền tảng của các chiến lược kinh doanh hiện đại.

Tuy nhiên, thách thức vẫn là trình bày lượng thông tin khổng lồ này theo cách dễ tiếp cận, hấp dẫn và dễ hiểu cho nhiều đối tượng khác nhau. Các phương pháp hiển thị truyền thống, chẳng hạn như bảng hiệu in hoặc màn hình kỹ thuật số, không phù hợp với môi trường có nhịp độ nhanh này. Các tài liệu in tuy quen thuộc về mặt hình ảnh nhưng lại phát sinh chi phí định kỳ do nhu cầu cập nhật và thay thế liên tục. Mặt khác, màn hình kỹ thuật số thông thường, mặc dù dễ cập nhật hơn nhưng thường tiêu tốn nhiều năng lượng và chi phí bảo trì. Việc chuyển đổi từ các phương pháp hiển thị truyền thống sang các giải pháp năng động, tiết kiệm chi phí và tiết kiệm năng lượng hơn không chỉ là một lựa chọn mà còn là điều cần thiết để bắt kịp bối cảnh kinh doanh đang phát triển nhanh chóng của thời đại kỹ thuật số.

Bằng cách kết hợp những tiến bộ mới nhất trong công nghệ mực điện tử với các nguyên tắc thiết kế tiết kiệm năng lượng, công nghệ hiển thị giấy điện tử đã nổi lên như một giải pháp mang tính biến đổi trong nhiều ngành công nghiệp khác nhau bên cạnh màn hình truyền thống và kỹ thuật số. Khả năng độc đáo của nó là giữ lại văn bản và hình ảnh mà không cần nguồn điện liên tục khiến nó trở nên lý tưởng cho nhiều ứng dụng, từ bảng hiệu động trong bán lẻ và vận chuyển đến hiển thị thông tin theo thời gian thực trong môi trường doanh nghiệp và chăm sóc sức khỏe, thậm chí còn cách mạng hóa việc cung cấp thông tin trong dịch vụ ngành công nghiệp. Ngoài ra, công nghệ này phù hợp với sự chú trọng ngày càng tăng vào tính bền vững môi trường và hiệu quả năng lượng trong ngành, mang đến cho doanh nghiệp một cách tiết kiệm chi phí để đón đầu xu hướng thông tin, đảm bảo rằng khách hàng có quyền truy cập vào dữ liệu mới nhất đồng thời giảm lượng khí thải carbon của họ xuống mức tối đa. đạt được các mục tiêu bền vững. Sự thay đổi này không chỉ đơn thuần là sự cải tiến mà còn là sự phát triển cần thiết để đáp ứng nhu cầu của môi trường dịch vụ hiện đại, nơi thông tin vừa là công cụ vừa là tài sản.

Dự án này nhằm mục đích tập trung vào việc xây dựng nguyên mẫu hệ thống hiển thị EPD triển khai việc sử dụng công nghệ hiển thị giấy điện tử ở nhiều khía cạnh khác nhau, bao gồm kho bãi, giáo dục, doanh nghiệp, v.v.

\section{Mục tiêu và phạm vi đề tài}
\label{section:1.2}
Có rất nhiều ứng dụng của giấy điện tử đang được sử dụng trong nhiều hệ thống lớn khác nhau, từ giao thông công cộng, hậu cần đến giáo dục. Tuy nhiên, đây là những hệ thống mở rộng yêu cầu xử lý nhiều nhiệm vụ chuyên biệt và yêu cầu hệ thống hiển thị phức tạp và tốn kém. Nói cách khác, hệ thống EPD (Electronic Paper Display) sử dụng trong các doanh nghiệp này được tối ưu hóa cao cho nhu cầu cụ thể của từng doanh nghiệp, khiến chi phí đắt đỏ và không phù hợp với các hệ thống nhỏ hơn.

Dựa trên những hạn chế này, dự án nhằm mục đích phát triển hệ thống hiển thị EPD tập trung nhiều hơn vào các doanh nghiệp và các trường hợp sử dụng nhỏ hơn, cũng như nhắm đến các mục tiêu chính sau: (i) Cung cấp cách hiển thị dữ liệu hiệu quả về mặt chi phí bằng cách sử dụng công nghệ hiển thị giấy điện tử ở quy mô nhỏ các doanh nghiệp; (ii) Cung cấp giao diện người dùng quản lý và hệ thống back-end phù hợp với hầu hết các nhu cầu của mọi khía cạnh, trước hết tập trung vào bán lẻ, văn phòng, bệnh viện, trường học và triển lãm; và (iii) Tối ưu hóa các thiết bị EPD để đạt được hiệu suất tốt hơn và chất lượng cao hơn trong nhiều môi trường khắc nghiệt khác nhau.

\section{Định hướng giải pháp}
\label{section:1.3}
Để đạt được các mục tiêu đặt ra tại Mục \ref{section:1.2}, dự án sẽ tập trung phát triển hệ thống theo mô hình client-server kết hợp với microservices để đảm bảo tính linh hoạt, dễ quản lý và có khả năng mở rộng khi cần thiết.

Về phía khách hàng, dự án sử dụng khung NextJS để phát triển giao diện người dùng thân thiện với người dùng và thư viện tùy chỉnh nhằm nhận và hiển thị dữ liệu trong các thiết bị EPD chạy trên mô-đun phát triển supermini ESP32-C3. Về mặt back-end, dự án sử dụng ExpressJS và MongoDB để xây dựng máy chủ API chứa dữ liệu người dùng và RabbitMQ làm Nhà môi giới MQTT để giao tiếp với các thiết bị EPD thông qua giao thức MQTT. Thông tin chi tiết về mô hình được sử dụng và các công nghệ được sử dụng trong hệ thống sẽ được trình bày cụ thể trong các Chương \ref{chapter:Methodology} và \ref{chaptef:Experiment}.

Dự án cũng sử dụng các kết nối bảo mật giữa các thành phần để bảo vệ dữ liệu người dùng trước các cuộc tấn công từ bên ngoài.
\section{Bố cục đồ án}
\label{section:1.4}
Phần còn lại của luận án này được cấu trúc như sau:

Chương 2 trình bày quá trình khảo sát và phân tích yêu cầu của dự án, cung cấp cái nhìn tổng quan về chức năng của dự án và trình bày chi tiết một số tính năng nổi bật của hệ thống hiển thị EPD.

Dựa trên các yêu cầu được nêu trong Chương 2, Chương 3 sẽ tập trung vào các công nghệ chính để phát triển hệ thống. Hệ thống sẽ bao gồm ba phần chính: (i) trang web quản lý thông tin và thiết bị hiển thị; (ii) máy chủ Node.js và Nhà môi giới MQTT để xử lý giao tiếp giữa các thiết bị hiển thị và máy chủ, cũng như xử lý các yêu cầu từ trang web; (iii) Thiết bị EPD để nhận và hiển thị thông tin.

Chương 4 sẽ đi sâu hơn vào thiết kế kiến trúc của từng thành phần, thiết kế cơ sở dữ liệu, quy trình phát triển ứng dụng, thử nghiệm và bảo mật hệ thống.

Chương 5 thảo luận về những đóng góp đáng kể và giải pháp nổi bật giúp hệ thống giải quyết các vấn đề hiện tại như quản lý thiết bị từ xa, truy xuất thông tin thiết bị có giá trị và bảo mật hệ thống. Các vấn đề hoặc thách thức còn lại phải đối mặt trong quá trình phát triển hệ thống cũng được thảo luận trong chương này.

Cuối cùng, Chương 6 kết thúc với những đóng góp chính của dự án và định hướng phát triển trong tương lai của hệ thống.
\end{document}