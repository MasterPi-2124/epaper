\documentclass[../Main.tex]{subfiles}
\begin{document}
Chương này có độ dài từ 9 đến 11 trang.

Với phương pháp phân tích thiết kế hướng đối tượng, sinh viên sử dụng biểu đồ use case theo hướng dẫn của template này. Với các phương pháp khác, sinh viên trao đổi với giáo viên hướng dẫn để đổi tên và sắp xếp lại đề mục cho phù hợp. Ví dụ, thay vì sử dụng biểu đồ use case, sinh viên đi theo hướng tiếp cận Agile có thể dùng User Story.
hhhh

\textbf{Lưu ý}: Mỗi chương nên có thêm 1 đoạn mở đầu chương và kết thúc chương, mở đầu giới thiệu những nội dung sẽ trình bày trong chương, kết thúc tổng kết lại các nội dung đã trình bày.

\section{Status survey}
\label{section:2.1}
Thông thường, khảo sát chi tiết về hiện trạng và yêu cầu của phần mềm sẽ được lấy từ ba nguồn chính, đó là (i) người dùng/khách hàng, (ii) các hệ thống đã có, (iii) và các ứng dụng tương tự.
Sinh viên cần tiến hành phân tích, so sánh, đánh giá chi tiết ưu nhược điểm của các sản phẩm/nghiên cứu hiện có. Sinh viên có thể lập bảng so sánh nếu cần thiết. Kết hợp với khảo sát người dùng/khách hàng (nếu có), sinh viên nêu và mô tả sơ lược các tính năng phần mềm quan trọng cần phát triển.

\section{Functional Overview}
\label{section:2.2}
Phần \ref{section:2.2} này có nhiệm vụ tóm tắt các chức năng của phần mềm. Trong phần này, sinh viên lưu ý chỉ mô tả chức năng mức cao (tổng quan) mà không đặc tả chi tiết cho từng chức năng. Đặc tả chi tiết được trình bày trong phần \ref{section:2.3}.

\subsection{General use case diagram}
\label{subsection:2.2.1}
Sinh viên vẽ biểu đồ use case tổng quan và giải thích các tác nhân tham gia là gì, nêu vai trò của từng tác nhân, và mô tả ngắn gọn các use case chính.

\subsection{Detailed use case diagram}
\label{subsection:2.2.2}
Với mỗi use case mức cao trong biểu đồ use case tổng quan, sinh viên tạo một mục riêng như mục \ref{subsection:2.2.2} và tiến hành phân rã use case đó. Lưu ý tên use case cần phân rã trong biểu đồ use case tổng quan phải khớp với tên đề mục.

Trong mỗi mục như vậy, sinh viên vẽ và giải thích ngắn gọn các use case phân rã.

\subsection{Business process}
\label{subsection:2.2.3}
Nếu sản phẩm/hệ thống cần xây dựng có quy trình nghiệp vụ quan trọng/đáng chú ý, sinh viên cần mô tả và vẽ biểu đồ hoạt động minh họa quy trình nghiệp vụ đó. Sinh viên lưu ý đây không phải là luồng sự kiện của từng use case, mà là luồng hoạt động kết hợp nhiều use case để thực hiện một nghiệp vụ nào đó.

Ví dụ, một hệ thống quản lý thư viện có quy trình nghiệp vụ mượn trả với mô tả sơ bộ như sau: Sinh viên làm thẻ mượn, sau đó sinh viên đăng ký mượn sách, thủ thư cho mượn, và cuối cùng sinh viên trả lại sách cho thư viện. Một hệ thống có thể có một vài quy trình nghiệp vụ quan trọng như vậy.
\section{Functional description}
\label{section:2.3}
Sinh viên lựa chọn từ 4 đến 7 use case quan trọng nhất của đồ án để đặc tả chi tiết. Mỗi đặc tả bao gồm ít nhất các thông tin sau: (i) Tên use case, (ii) Luồng sự kiện (chính và phát sinh), (iii) Tiền điều kiện, và (iv) Hậu điều kiện. Sinh viên chỉ vẽ bổ sung biểu đồ hoạt động khi đặc tả use case phức tạp.
\subsection{Description of use case A}
\hfill
\subsection{Description of use case B}
\hfill

\section{Non-functional requirement}
\label{section:2.4}
Trong phần này, sinh viên đưa ra các yêu cầu khác nếu'k có, ghghjghjgbao gồm các yêu cầu phi chức năng như hiệu năng, độ tjkl;kjl;jkl;in cậy, tính dễ dùng, tính dễ bảo trì, hoặc các yêu cầu về mặt kỹ thuật như về CSDL, công nghệ sử dụng, v.v.


%%%%%%%%%%%%%%%%%%%%%%%%%%%%%%%%%%%

\end{document}