
\documentclass[../Main.tex]{subfiles}
\begin{document}
Based on the real-world problems discussed in Chapter 1, Chapter 2 aims to delve deeper into the existing products and solutions that can help address these issues. Specifically, it will also analyze some detailed use cases of the Data Management System, as outlined in Section 2.2.
\section{Status survey}
\label{section:2.1}
Thông thường, khảo sát chi tiết về hiện trạng và yêu cầu của phần mềm sẽ được lấy từ ba nguồn chính, đó là (i) người dùng/khách hàng, (ii) các hệ thống đã có, (iii) và các ứng dụng tương tự.
Sinh viên cần tiến hành phân tích, so sánh, đánh giá chi tiết ưu nhược điểm của các sản phẩm/nghiên cứu hiện có. Sinh viên có thể lập bảng so sánh nếu cần thiết. Kết hợp với khảo sát người dùng/khách hàng (nếu có), sinh viên nêu và mô tả sơ lược các tính năng phần mềm quan trọng cần phát triển.

\section{Functional Overview}
\label{section:2.2}
% Phần \ref{section:2.2} này có nhiệm vụ tóm tắt các chức năng của phần mềm. Trong phần này, sinh viên lưu ý chỉ mô tả chức năng mức cao (tổng quan) mà không đặc tả chi tiết cho từng chức năng. Đặc tả chi tiết được trình bày trong phần \ref{section:2.3}.

\subsection{General use case diagram}
\label{subsection:2.2.1}
The general use case diagram of Data and EPD Devices Management System is illustrated in Figure 1. As per the diagram, the system involves three main agents, namely The Manager, The Administrator, and the EPD devices. The Manager has the ability to manage EPD devices, data and their own accounts. The Administrator inherits the functions of the Manager and can also manage and test the devices more advanced. On the other hand, the EPD devices act as end-users and receive and display data on the screen. They also interact with the system via MQTT protocol.
\begin{figure}[htbp]
    \centering
    \begin{tikzpicture}
        % Define system boundary
        \umlactor[x=0, y=0]{Administrator}
        \umlactor[x=0, y=-4]{Manager}
        \umlactor[x=11.5, y=0]{EPD Devices}
        
        \begin{umlsystem}[x=4, y=0]{System}
            % Define the use cases
            \umlusecase[x=2, y=0, name=usecase1]{Manage personal accounts}
            \umlusecase[x=2, y=-2, name=usecase2]{Manage devices}
            \umlusecase[x=2, y=-4, name=usecase3]{Manage data}
        \end{umlsystem}
        
        % Draw associations
        \umlassoc{Administrator}{usecase1}
        \umlassoc{Administrator}{usecase2}
        \umlassoc{Manager}{usecase2}
        \umlassoc{Manager}{usecase3}
        \umlassoc{Manager}{usecase1}
        \umlassoc{EPD Devices}{usecase1}
        \umlinherit{Administrator}{Manager} 
        
        \draw (-1.5, 2.5) rectangle (13, -6);
    \end{tikzpicture}
    \caption{General use case diagram of the data management system}
    \label{fig:usecasediagram}
\end{figure}

\subsection{Detailed use case diagram}
\label{subsection:2.2.2}

\subsubsection{Detailed use case of User's Account Management function}

Figure 2 below describes the detailed use case diagram of the User's Account Management function, including the Manager and Administrator. Users can create a new account, log in, and modify personal information.

\begin{figure}[htbp]
    \centering
    \begin{tikzpicture}
        \umlactor[x=0, y=0]{Administrator}
        \umlactor[x=0, y=-4]{Manager}
    
        \begin{umlsystem}[x=4, y=0]{System}
            % Define the use cases
            \umlusecase[x=2, y=0, name=usecase1]{Register}
            \umlusecase[x=2, y=-2, name=usecase2]{Log in}
            \umlusecase[x=2, y=-4, width=3cm, name=usecase3]{Manage personal information}
        \end{umlsystem}
        
        % Draw associations
        \umlassoc{Manager}{usecase1}
        \umlassoc{Manager}{usecase2}
        \umlassoc{Manager}{usecase3}
        \umlinherit{Administrator}{Manager} 
        
        \draw (-1.5, 2.5) rectangle (13, -6);
    \end{tikzpicture}
    \caption{Detailed use case of User's Account Management function}
    \label{fig:usecasediagram}
\end{figure}

\subsubsection{Detailed use case of Data Management function}
A detailed use case diagram of the Data Management function of the Manager is displayed in Figure 3. This function enables the Manager to view a list of data, add, modify, or remove data information, and choose whether to display the data on the device or not.

\begin{figure}[htbp]
    \centering
    \begin{tikzpicture}
        \umlactor[x=0, y=-2]{Manager}
        \umlactor[x=11.5, y=-1]{EPD Devices}
        
        \begin{umlsystem}[x=4, y=0]{System}
            % Define the use cases
            \umlusecase[x=2, y=0, name=usecase1]{Add displaying data}
            \umlusecase[x=2, y=-2, name=usecase2]{Modify data information}
            \umlusecase[x=2, y=-4, width=3cm, name=usecase3]{Display data on device}
            \umlusecase[x=2, y=-6, width=3cm, name=usecase4]{Remove data}
        \end{umlsystem}
        
        % Draw associations
        \umlassoc{Manager}{usecase1}
        \umlassoc{Manager}{usecase2}
        \umlassoc{Manager}{usecase3}
        \umlassoc{Manager}{usecase4}
        \umlassoc{EPD Devices}{usecase2}
        \umlassoc{EPD Devices}{usecase3}
        
        \draw (-1.5, 2.5) rectangle (13, -8);
    \end{tikzpicture}
    \caption{Detailed use case of Data Management function}
    \label{fig:usecasediagram}
\end{figure}

\subsection{Business process}
\label{subsection:2.2.3}


\section{Functional description}
\label{section:2.3}
Sinh viên lựa chọn từ 4 đến 7 use case quan trọng nhất của đồ án để đặc tả chi tiết. Mỗi đặc tả bao gồm ít nhất các thông tin sau: (i) Tên use case, (ii) Luồng sự kiện (chính và phát sinh), (iii) Tiền điều kiện, và (iv) Hậu điều kiện. Sinh viên chỉ vẽ bổ sung biểu đồ hoạt động khi đặc tả use case phức tạp. ok kj ghj
\subsection{Description of use case A}
\hfill
\subsection{Description of use case B}
\hfill

\section{Non-functional requirement}
\label{section:2.4}
Trong phần này, sinh viên đưa ra các yêu cầu khác nếu'ck có, gjkkhghjghjgbao gồm các yêu cầu phi chức năng như hiệucsdfsdfsdfcccc năng, độ tjkl;kjl;jkl;in cậy, tính dễ dùng, tính dễ bảo trì, hoặc các yêu cầu về mặt kỹ thuật như về CSDL, công nghệ sử dụng, v.v.
sdfsdfasd asda sdasd dgfdfg

%%%%%%%%%%%%%%%%%%%%%%%%%%%%%%%%%%%

\end{document}