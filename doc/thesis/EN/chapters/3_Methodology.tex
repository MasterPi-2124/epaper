\documentclass[../Main.tex]{subfiles}
\begin{document}
Based on the real-life surveyed situation and the detailed system analysis in Chapter 2, Chapter 3 will center on the technologies used to develop the Data Management System. This system comprises three main components, namely, the Management Frontend, Backend, and the EPD Devices that are part of the system. The Management UI is built using NextJS and TailwindCSS to provide an easy-to-use dashboard for managing multiple devices and data efficiently. The UI also incorporates protocols to connect to HTTP, MQTT servers, and EPD devices via a Serial port. All data is processed and stored on the server using MongoDB and ExpressJS. The server communicates with EPD devices through the MQTT protocol, with RabbitMQ acting as a broker and SSL/TLS ensuring a secure connection. The EPD devices connect to the broker and subscribe to a specific topic to receive and handle any information requested by users from the front-end side, as well as send the status back to the broker.

\section{Management UI (Front-end side)}
\subsection{NextJS}

Note: because of its ..., NextJS help reduce an enourmous amount of additional third-party components, which helps creating the UI in an efficient way and also help speeding up the UI's overall responsive time.
\subsection{TailwindCSS}

because of its strength, TailwindCSS helps reducing a lot of redundant css syntax by implement it right into class, which minify JS packets when the website is rendering

\section{Server (Back-end side)}
Because of the complexity of the system that needs both processing data from client side, and also communicating with EPD devices, this Server section is divided into 2 main parts: API server handling user's requests and storing data in MongoDB; and RabbitMQ broker communicating with devices and sending data back to ExpressJS server to process. The server used is a dedicated server hosted in Hetzner, which is a server company located in Europe, with quality is ensured to be among of the best in the world. This dedicated server also ensures the privacy, security and durable span.

\subsection{API Server}
This part of the back-end server is in charge of receiving user requests, managing data to the MQTT Server, and storing data in the system. The server is written in MVCS design pattern for code transparency and management and leverages \textbf{ExpressJS}, a minimal and flexible Node.js web application framework, to help enhance the server's capabilities. It enables efficient API creation and middleware integration, streamlining development and improving maintainability.

To store and secure user data, the back-end system uses \textbf{MongoDB Community Edition for Linux} (hereafter referred to as '\textbf{MongoDB}'), which stands out from other database solutions due to its distinct advantages in storing and processing data. MongoDB is adept at handling vast amounts of user and device data, storing it securely across distributed systems, in this case, the \textit{clusters}. This design enhances data redundancy and security and allows for horizontal scaling, accommodating large-scale data without compromising on system performance. In tandem with \textbf{ExpressJS}, this also ensures efficient, high-performance query and transaction processing in a timely manner while keeping the development and maintenance processes streamlined and minimal.

Utilized for API design and documentation in this project, Swagger from OpenAPI, known as OpenAPI Specification (OAS), offers a comprehensive, detailed description of the API structure, including endpoints, operations, and parameters. This documentation tool not only aids in API development and testing by allowing for interaction with API resources without implementing logic, but it also streamlines client SDK generation across various programming languages. Leveraging Swagger documentation in the project helps speed up the development cycle and enhance efficiency in identifying bugs and issues during testing, ultimately resulting in a more robust and user-friendly application.

\subsection{MQTT Broker}
To enable real-time communication and data transfer between users and devices, the system also operates as an MQTT broker, facilitating the exchange of information between parties. This is acquired by using RabbitMQ, an open-source message broker software that enables applications to communicate with each other and exchange information efficiently. It acts as an intermediary for messaging by accepting and forwarding messages, making it a critical tool for handling asynchronous communication between EPD devices and the server. . Its reliability and scalability make it a preferred choice for enterprises needing to ensure message delivery without loss, even in high-throughput scenarios. RabbitMQ's ability to decouple processes also leads to more resilient and manageable application architectures.
\end{document}




% obey shell ski task neglect bring stable crunch stable all offer rely