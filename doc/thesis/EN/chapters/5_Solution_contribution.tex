\documentclass[../Main.tex]{subfiles}
\begin{document}

Chương này có độ dài tối thiểu 5 trang, tối đa không giới hạn.\footnote{Trong trường hợp phần này dưới 5 trang thì sinh viên nên gộp vào phần kết luận, không tách ra một chương riêng rẽ nữa.} Sinh viên cần trình bày tất cả những nội dung đóng góp mà mình thấy tâm đắc nhất trong suốt quá trình làm ĐATN. Đó có thể là một loạt các vấn đề khó khăn mà sinh viên đã từng bước giải quyết được, là giải thuật cho một bài toán cụ thể, là giải pháp tổng quát cho một lớp bài toán, hoặc là mô hình/kiến trúc hữu hiệu nào đó được sinh viên thiết kế.

Chương này \textbf{là cơ sở quan trọng} để các thầy cô đánh giá sinh viên. Vì vậy, sinh viên cần phát huy tính sáng tạo, khả năng phân tích, phản biện, lập luận, tổng quát hóa vấn đề và tập trung viết cho thật tốt.
Mỗi giải pháp hoặc đóng góp của sinh viên cần được trình bày trong một mục độc lập bao gồm ba mục con: (i) dẫn dắt/giới thiệu về bài toán/vấn đề, (ii) giải pháp, và (iii) kết quả đạt được (nếu có).

Sinh viên lưu ý \textbf{không trình bày lặp lại nội dung}. Những nội dung đã trình bày chi tiết trong các chương trước không được trình bày lại trong chương này. Vì vậy, với nội dung hay, mang tính đóng góp/giải pháp, sinh viên chỉ nên tóm lược/mô tả sơ bộ trong các chương trước, đồng thời tạo tham chiếu chéo tới đề mục tương ứng trong Chương 5 này. Chi tiết thông tin về đóng góp/giải pháp được trình bày trong mục đó.

Ví dụ, trong Chương 4, sinh viên có thiết kế được kiến trúc đáng lưu ý gì đó, là sự kết hợp của các kiến trúc MVC, MVP, SOA, v.v. Khi đó, sinh viên sẽ chỉ mô tả ngắn gọn kiến trúc đó ở Chương 4, rồi thêm các câu có dạng: ``Chi tiết về kiến trúc này sẽ được trình bày trong phần 5.1". 


\end{document}