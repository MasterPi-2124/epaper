\documentclass[../Main.tex]{subfiles}
\begin{document}

The preceding chapters of this document have meticulously delineated the architectural framework and the technological foundations employed in the development of the system. However, without any distinct solutions and initiatives, the system might have remained indistinguishable within the extensive landscape of business solutions, and not suitable in other environment like Vietnam. This chapter aims to shed light on how the issues are identified in the development progress of the project, and the personalized strategies and innovations that contributed to the system's uniqueness and also enhanced its efficacy and relevance in a dynamic business environment.

\section{Display custom text}
In the development progress, the system comes to an issue when user wants to display Vietnamse language texts and other special characters that are not in ASCII character map. Vietnamese, with its complex diacritical marks and distinct character set, demands a high degree of precision in rendering, a task that is particularly challenging given the inherent limitations and operational mechanics of e-paper displays. This section will focus on how a normal text is store and process in 

\subsection{UTF-8 vs. UTF-16}
UTF-8 and UTF-16 are both encoding formats used for representing text in computers, each with its unique handling of character sets, including special characters like those in the Vietnamese language. Different from ASCII, where each character uses one byte to store, UTF-8 uses one to four bytes per character, making it highly capable of 1112064 valid character codes while still being backward-compatible with ASCII's old code. On the other hand, UTF-16 uses fixed-length two or four bytes per character, offering a balance between the space efficiency of UTF-8 and the simplicity of fixed-length encoding. In the scope of the project, both encoding formats are compatible with processing Vietnamese characters to display on screen, each with its unique strengths and weaknesses.

\subsection{Two ways of process and display text}
In order to display text to the screen, the text has to be converted to bitmap image, stored in a form of  an array of bytes. This process can be 
\subsection{Looping in font map}
\subsection{Shapes and Bitmap Images}
\subsection{Partial Display}
a
\section{ESP32 Optimization}
\subsection{Serial Port}
\subsection{Serial Port}

\section{RabbitMQ}

\section{Security and vulnerabilities}
\subsection{MongoDB Hack}
\subsection{TLS/SSL}


% Chương này có độ dài tối thiểu 5 trang, tối đa không giới hạn.\footnote{Trong trường hợp phần này dưới 5 trang thì sinh viên nên gộp vào phần kết luận, không tách ra một chương riêng rẽ nữa.} Sinh viên cần trình bày tất cả những nội dung đóng góp mà mình thấy tâm đắc nhất trong suốt quá trình làm ĐATN. Đó có thể là một loạt các vấn đề khó khăn mà sinh viên đã từng bước giải quyết được, là giải thuật cho một bài toán cụ thể, là giải pháp tổng quát cho một lớp bài toán, hoặc là mô hình/kiến trúc hữu hiệu nào đó được sinh viên thiết kế.

% Chương này \textbf{là cơ sở quan trọng} để các thầy cô đánh giá sinh viên. Vì vậy, sinh viên cần phát huy tính sáng tạo, khả năng phân tích, phản biện, lập luận, tổng quát hóa vấn đề và tập trung viết cho thật tốt.
% Mỗi giải pháp hoặc đóng góp của sinh viên cần được trình bày trong một mục độc lập bao gồm ba mục con: (i) dẫn dắt/giới thiệu về bài toán/vấn đề, (ii) giải pháp, và (iii) kết quả đạt được (nếu có).

% Sinh viên lưu ý \textbf{không trình bày lặp lại nội dung}. Những nội dung đã trình bày chi tiết trong các chương trước không được trình bày lại trong chương này. Vì vậy, với nội dung hay, mang tính đóng góp/giải pháp, sinh viên chỉ nên tóm lược/mô tả sơ bộ trong các chương trước, đồng thời tạo tham chiếu chéo tới đề mục tương ứng trong Chương 5 này. Chi tiết thông tin về đóng góp/giải pháp được trình bày trong mục đó.

% Ví dụ, trong Chương 4, sinh viên có thiết kế được kiến trúc đáng lưu ý gì đó, là sự kết hợp của các kiến trúc MVC, MVP, SOA, v.v. Khi đó, sinh viên sẽ chỉ mô tả ngắn gọn kiến trúc đó ở Chương 4, rồi thêm các câu có dạng: ``Chi tiết về kiến trúc này sẽ được trình bày trong phần 5.1". 


\end{document}