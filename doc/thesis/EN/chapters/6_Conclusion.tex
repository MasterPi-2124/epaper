\documentclass[../Main.tex]{subfiles}
\begin{document}
\section{Conclusion}
Recognizing the need for an effective way of displaying information in many use cases, the thesis \textbf{"Building Information board using ePaperboard"} has developed into a comprehensive study that not only aims to provide solutions to the problem but also contributes valuable insights and solutions. The research process, together with the design and development processes, are also discussed in this thesis. Although the project has built an EPD system with core functions, it still has shortcomings and limitations that need to be overcome and directions that will be implemented in the future.

\subsection{Achievements}
Thanks to the support from MSc. Nguyen Duc Tien, the project has achieved the goals and objectives discussed in previous chapters. First, the project has researched the need for effective displaying in various business environments, together with four leading EPD manufacturers \textbf{E Ink}, \textbf{Seekink}, \textbf{Inkcase}, and \textbf{Pervasive Displays}, evaluated their uniqueness and weaknesses to pinpoint the objective and essential features of the EPD devices and the management system. Second, the project has delved deeper into analyzing the functional and non-functional requirements, hence designing the details of the management system. From these designs, the project has built EPD devices that fulfilled all the requirements and the management UI for the user to manage devices and data. The system has also been tested in various example environments and proved helpful in multiple use cases.

One of the significant findings of this study is Vietnamese font optimization. This finding not only contributes to the understanding of displaying on the EPD panel and resource management in ESP32-C3 but also has practical implications in custom display on the EPD devices.

\subsection{Knowledge and Experiences}
Throughout the process of researching, developing, and testing my project, I have acquired a wide range of knowledge, skills, and valuable experiences. Firstly, I utilized my knowledge from the subjects "Introduction to Software Engineering" and "Information System Analysis and Design" to gain a clear understanding of the system and create a detailed plan. Secondly, by utilizing various techniques and knowledge gained from other subjects, I was able to develop EPD devices and a management system, as well as optimize the entire system. Finally, after conducting market research and analyzing competitors, I realized the importance of focusing on the right aspects of the project. Without a thorough understanding of the market and competition, the system could end up being just one of many similar products that fail to solve the existing problems.

\subsection{Shortcomings and limitations}
Despite its achievements, the project still has several significant shortcomings and unresolved issues that need to be addressed. Firstly, the battery life of the EPD devices remains poor even after optimization due to the limited capacity of the included battery. The ESP32-C3 is continuously running to maintain a connection to WiFi and MQTT broker, which further drains the battery. Secondly, the system is currently in the prototype stage, which means it is only suitable for small-scale use cases. However, in a real business environment, it will need to be improved significantly to meet the extensive needs of the business. Lastly, the management UI is not optimized for several screen sizes, making it challenging to use in various scenarios.

\section{Future work}

The system's current functions have achieved the project's goals, but there are still valuable features that need to be further developed to provide users with a better experience.

First, the system will need a better combined way of managing data and devices, as discussed in section \ref{opt} in chapter \ref{chapter:SolutionAndContribution}. This feature will enable the system to accommodate many other specific use cases, enhancing scalability and flexibility. 

Second, the EPD device will need to be re-designed for optimal battery life and visual appeal. Additionally, the ESP32-C3 should be more energy-efficient while still ensuring performance. Adding different types of EPD devices, such as different sizes and multi-colors, is also encouraged to provide device diversity in the system.

Third, the users in the system should have a more apparent hierarchy and rights. A specific decentralized solution that the system plans to develop in the future is creating admin-type users who are responsible for managing the system. The users of that type can manage their own MongoDB and RabbitMQ servers with custom accounts. They can also manage multiple other users and assign roles to specific users. Furthermore, they can also control the devices and data of all managed users but can not add, remove, or modify them. This hierarchal solution is beneficial in large businesses when a company needs to handle a lot of sub-departments and data, and an effective way of managing users and data is crucial.

Fourth, the Management UI needs to be optimized for multiple screen sizes and mobile-friendly. A mobile application is also highly recommended to broaden the use of the system across different devices, users, and use cases.

Above is the entire content of the project \textbf{"Building Information board using ePaperboard"}. In general, the project has provided valuable insights into the practical application of electrophoretic display in various use cases and offered a solid foundation for further exploration and development in intelligent display. With limited time, skills, and experience, it was challenging to avoid shortcomings during the project development process. I look forward to receiving guidance and comments from teachers so that the system can be better in the future.
\end{document}