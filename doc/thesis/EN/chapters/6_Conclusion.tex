\documentclass[../Main.tex]{subfiles}
\begin{document}
\section{Conclusion}
Recognizing the need for an effective way of displaying information in many use cases, the thesis \textbf{"Building Information board using ePaperboard"} has developed into a comprehensive study that not only aims to provide solutions to the problem but also contributes valuable insights and solutions. The research process, together with the design and development processes, are also discussed in this thesis. Although the project has built an EPD system with core functions, it still has shortcomings and limitations that need to be overcome and directions that will be implemented in the future

\subsection{Achievements}
Thanks to the support from MSc. Nguyen Duc Tien, the project has achieved the goals and objectives discussed in previous chapters. First, the project has researched the need for effective displaying in various business environments, together with four leading EPD manufacturers \textbf{E Ink}, \textbf{Seekink}, \textbf{Inkcase}, and \textbf{Pervasive Displays}, evaluated their uniqueness and weaknesses to pinpoint the objective and essential features of the EPD devices and the management system. Second, the project has delved deeper into analyzing the functional and non-functional requirements, hence designing the details of the management system. From these designs, the project has built EPD devices meeting all the requirements and the management UI for the user to manage devices and data. The system has also been tested in various example environments and proved useful in multiple use cases.

One of the significant findings of this study is Vietnamese font optimization. This finding not only contributes to the understanding of displaying on the EPD panel and resource management in ESP32-C3 but also has practical implications in custom display on the EPD devices.

\subsection{Knowledge and Experiences}
In the researching, developing, and testing process of the project, I have gained and learned a lot of knowledge and skills and accumulated many valuable experiences for myself. Firstly, using the knowledge learned from "(Introduction to Software Engineering)" and "Information System Analysis and Design" subjects, I can have a clear insight into the system and can organize and design a detailed system. Secondly, with the techniques and knowledge learned in various subjects, I can optimize. Finally, after surveying the market and other similar product providers, I realized the importance of understanding what the project should focus on, and if not having throughout surveying, the system could be one of several other same products in the markets and still could not solve the existing problem. 

\subsection{Shortcomings and limitations}
Besides the achievements, the project still reveals many shortcomings and unresolved problems. First, the battery life of the EPD devices is still poor after optimization. This is due to the limited capability of the included battery, and ESP32-C3 is constantly running to maintain connection to WiFi and MQTT broker. Second, the system is still in the prototype stage, meaning it is only suitable for small use cases. In the actual business environment, it will need to improve to be capable of the business's extensive needs. Finally, the management UI is not optimized for many screen sizes, making it difficult to use in various cases.

\section{Future work}
The functions provided by the system have met the goals of the project; however, there are still some valuable features that need to be further implemented to bring users a better experience.

First, the system will need a better combined way of managing data and devices, as discussed in section \ref{opt} in chapter \ref{chapter:SolutionAndContribution}. This feature will enable the system to fit many other specific use cases, enhancing scalability and flexibility. 

Second, the EPD device will need to be re-designed for optimal battery life and visual look. Also, the ESP32-C3 will need to be more energy-efficient while still ensuring performance. Additional types of EPD devices are also encouraged to be added, from different sizes to multi-colors, etc. 

Third, the users in the system should have a clearer hierarchy and rights. A specific decentralized solution that the system plans to develop in the future is creating admin-type users who are responsible for managing the system. The users of that type can manage their own MongoDB and RabbitMQ servers with custom accounts. They can also manage multiple other users, create, modify, remove, or assign roles to specific users. Additionally, they can also manage the devices and data of all managed users but can not add, remove, or modify them. This hierarchal solution is very useful in large businesses when a company needs to manage a lot of sub-departments and data, and an effectve way of users and data management is crucial.

Fourth, the Management UI should be optimized for multple screen size, and mobile-friendly. A mobile application is also highly recommended to broaden the use of the system in different devices, users and use cases.



\end{document}


Additionally, this research has opened up new avenues for [talk about any new questions or areas your research has uncovered]. Future st dies might focus on [suggest areas for further research, building on your work].

Overall, the findings of this research provide valuable insights into [summarize the overall significance of your research in your field]. While the e are limitations such as [mention any limitations of your study], the study offers a solid foundation for further exploration and development in [mention the field or topic].

In conclusion, this thesis represents a significant step towards [restate the main goal or purpose of your research]. It contri utes to the existing body of knowledge by [summarize how your research contributes to the existing knowledge], thereby enhancing our understanding and offering new perspectives on [mention the broader topic or field].


